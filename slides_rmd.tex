\documentclass[,aspectratio=43]{beamer}

\usepackage{empheq}
\usepackage{ragged2e}

% ------------------------------------------------------------------------------
% Load theme -------------------------------------------------------------------
% ------------------------------------------------------------------------------
\usetheme[banner,logo]{ubd}

% Information for the title page -----------------------------------------------
\author{Haziq Jamil}

\title{UBD Beamer Theme using RMarkdown}

\title{UBD Beamer Theme using RMarkdown}

\subtitle{An example presentation document with R code}

\institute{Mathematical Sciences, Faculty of Science, UBD\\
\url{https://haziqj.ml}}

\date{\today}

% Font fix ---------------------------------------------------------------------
% \usepackage{ifxetex,ifluatex}
% \ifnum 0\ifxetex 1\fi\ifluatex 1\fi=0 % if pdftex
%   \usepackage[T1]{fontenc}
%   \usepackage[utf8]{inputenc}
%   \usepackage{textcomp} % provide euro and other symbols
% \else % if luatex or xetex
%   \usepackage{unicode-math}
%   \defaultfontfeatures{Scale=MatchLowercase}
%   \defaultfontfeatures[\rmfamily]{Ligatures=TeX,Scale=1}
% \fi

\usepackage{soul}
\makeatletter
\let\HL\hl
\renewcommand\hl{%
  \let\set@color\beamerorig@set@color
  \let\reset@color\beamerorig@reset@color
  \HL}
\makeatother
% https://tex.stackexchange.com/questions/460731/highlight-color-a-part-of-text-in-block-in-beamer
\newcommand{\hlc}[2][yellow]{{%
    \colorlet{foo}{#1}%
    \sethlcolor{foo}\hl{#2}}%
}
% https://tex.stackexchange.com/questions/352956/how-to-highlight-text-with-an-arbitrary-color

% knitr stuff ------------------------------------------------------------------
\usepackage{color}
\usepackage{fancyvrb}
\newcommand{\VerbBar}{|}
\newcommand{\VERB}{\Verb[commandchars=\\\{\}]}
\DefineVerbatimEnvironment{Highlighting}{Verbatim}{commandchars=\\\{\}}
% Add ',fontsize=\small' for more characters per line
\usepackage{framed}
\definecolor{shadecolor}{RGB}{248,248,248}
\newenvironment{Shaded}{\begin{snugshade}}{\end{snugshade}}
\newcommand{\AlertTok}[1]{\textcolor[rgb]{0.94,0.16,0.16}{#1}}
\newcommand{\AnnotationTok}[1]{\textcolor[rgb]{0.56,0.35,0.01}{\textbf{\textit{#1}}}}
\newcommand{\AttributeTok}[1]{\textcolor[rgb]{0.77,0.63,0.00}{#1}}
\newcommand{\BaseNTok}[1]{\textcolor[rgb]{0.00,0.00,0.81}{#1}}
\newcommand{\BuiltInTok}[1]{#1}
\newcommand{\CharTok}[1]{\textcolor[rgb]{0.31,0.60,0.02}{#1}}
\newcommand{\CommentTok}[1]{\textcolor[rgb]{0.56,0.35,0.01}{\textit{#1}}}
\newcommand{\CommentVarTok}[1]{\textcolor[rgb]{0.56,0.35,0.01}{\textbf{\textit{#1}}}}
\newcommand{\ConstantTok}[1]{\textcolor[rgb]{0.00,0.00,0.00}{#1}}
\newcommand{\ControlFlowTok}[1]{\textcolor[rgb]{0.13,0.29,0.53}{\textbf{#1}}}
\newcommand{\DataTypeTok}[1]{\textcolor[rgb]{0.13,0.29,0.53}{#1}}
\newcommand{\DecValTok}[1]{\textcolor[rgb]{0.00,0.00,0.81}{#1}}
\newcommand{\DocumentationTok}[1]{\textcolor[rgb]{0.56,0.35,0.01}{\textbf{\textit{#1}}}}
\newcommand{\ErrorTok}[1]{\textcolor[rgb]{0.64,0.00,0.00}{\textbf{#1}}}
\newcommand{\ExtensionTok}[1]{#1}
\newcommand{\FloatTok}[1]{\textcolor[rgb]{0.00,0.00,0.81}{#1}}
\newcommand{\FunctionTok}[1]{\textcolor[rgb]{0.00,0.00,0.00}{#1}}
\newcommand{\ImportTok}[1]{#1}
\newcommand{\InformationTok}[1]{\textcolor[rgb]{0.56,0.35,0.01}{\textbf{\textit{#1}}}}
\newcommand{\KeywordTok}[1]{\textcolor[rgb]{0.13,0.29,0.53}{\textbf{#1}}}
\newcommand{\NormalTok}[1]{#1}
\newcommand{\OperatorTok}[1]{\textcolor[rgb]{0.81,0.36,0.00}{\textbf{#1}}}
\newcommand{\OtherTok}[1]{\textcolor[rgb]{0.56,0.35,0.01}{#1}}
\newcommand{\PreprocessorTok}[1]{\textcolor[rgb]{0.56,0.35,0.01}{\textit{#1}}}
\newcommand{\RegionMarkerTok}[1]{#1}
\newcommand{\SpecialCharTok}[1]{\textcolor[rgb]{0.00,0.00,0.00}{#1}}
\newcommand{\SpecialStringTok}[1]{\textcolor[rgb]{0.31,0.60,0.02}{#1}}
\newcommand{\StringTok}[1]{\textcolor[rgb]{0.31,0.60,0.02}{#1}}
\newcommand{\VariableTok}[1]{\textcolor[rgb]{0.00,0.00,0.00}{#1}}
\newcommand{\VerbatimStringTok}[1]{\textcolor[rgb]{0.31,0.60,0.02}{#1}}
\newcommand{\WarningTok}[1]{\textcolor[rgb]{0.56,0.35,0.01}{\textbf{\textit{#1}}}}
\usepackage{graphicx,grffile}
\makeatletter
\def\maxwidth{\ifdim\Gin@nat@width>\linewidth\linewidth\else\Gin@nat@width\fi}
\def\maxheight{\ifdim\Gin@nat@height>\textheight\textheight\else\Gin@nat@height\fi}
\makeatother
% Scale images if necessary, so that they will not overflow the page
% margins by default, and it is still possible to overwrite the defaults
% using explicit options in \includegraphics[width, height, ...]{}
\setkeys{Gin}{width=\maxwidth,height=\maxheight,keepaspectratio}
% Set default figure placement to htbp
\makeatletter
\def\fps@figure{htbp}
\makeatother
\setlength{\emergencystretch}{3em} % prevent overfull lines
\providecommand{\tightlist}{%
  \setlength{\itemsep}{0pt}\setlength{\parskip}{0pt}}
\setcounter{secnumdepth}{-\maxdimen} % remove section numbering

% ------------------------------------------------------------------------------
% Packages ---------------------------------------------------------------------
% ------------------------------------------------------------------------------

% \setlength{\parskip}{1em}
\usepackage{tikz}
\usetikzlibrary{shapes.geometric,fit,arrows.meta}
\usepackage{xltabular}
\usepackage{longtable,booktabs,multirow,multicol,colortbl}

\usepackage{caption}
% Make caption package work with longtable
\makeatletter
\def\fnum@table{\tablename~\thetable}
\makeatother

\usepackage{lipsum}
\usepackage{csquotes}

% To use arabic ----------------------------------------------------------------
% WARNING: Using arabic script causes some issues with footnotes.
% Packages are not loaded by default
\usepackage{polyglossia}  
\setdefaultlanguage{english}
\setotherlanguage{arabic} % to use arabic
\newfontfamily\arabicfontsf[Script=Arabic]{Amiri}
\usepackage{xeCJK}
\setCJKmainfont{SimSun}
% \usepackage{fontspec}

% % Fix for footnotes not showing when arabic script used
% % https://tex.stackexchange.com/questions/228075/beamer-in-arabic-language-doesnt-accept-footnotes
% \makeatletter
% \let\@footnotetext=\beamer@framefootnotetext
% \makeatother

% % Fix for footnotes not showing when using \footnote<.->
% \let\oldfootnote\footnote
% \renewcommand{\footnote}{\only<+->\oldfootnote}
% % https://stackoverflow.com/questions/62345074/show-footnote-only-after-a-pause-in-beamer-with-r-markdown

% Fonts ------------------------------------------------------------------------
\usepackage{cmbright}
\usefonttheme{default}
\usepackage{pifont}% http://ctan.org/pkg/pifont
\newcommand{\cmark}{\ding{51}}%
\newcommand{\xmark}{\ding{55}}%

% Bibliography -----------------------------------------------------------------
\usepackage[style=authoryear,giveninits=true,maxcitenames=2,maxbibnames=99,backend=biber,natbib]{biblatex}
\renewcommand*{\mkbibacro}[1]{#1}
\addbibresource{refs.bib}

% Fix URL, DOI, ISBN, etc. font in biblatex
% https://tex.stackexchange.com/questions/416093/change-font-of-the-word-url-before-the-actual-url-in-biblatex
% \renewcommand*{\mkbibacro}[1]{#1}  

\newcommand{\thankyou}{%
	{
		\begin{frame}[plain,noframenumbering]{End}
			\centering
			\Huge Thank you!
		\end{frame}
	}

}

% ------------------------------------------------------------------------------
% Mathematics ------------------------------------------------------------------
% ------------------------------------------------------------------------------

\newcommand*\mybox[1]{%
\colorbox{navyblue!35}{#1}}

\usepackage[skins,theorems]{tcolorbox}
\tcbset{highlight math style={enhanced,
  colframe=navyblue,colback=white,arc=2pt,boxrule=1pt}}

\usepackage{amssymb}
\usepackage{dsfont}  % for indicator variables \mathsds{1}
\usepackage{bm}  % for better bold script
\usepackage[makeroom]{cancel}
\usepackage{centernot}
\renewcommand{\CancelColor}{\color{gray}}
\newcommand{\bzero}{{\bm 0}}
\newcommand{\bone}{{\bm 1}}
\newcommand{\ba}{{\bm a}}
\newcommand{\bb}{{\bm b}}
\newcommand{\bc}{{\bm c}}
\newcommand{\bd}{{\bm d}}
\newcommand{\be}{{\bm e}}
\newcommand{\bff}{{\bm f}}
\newcommand{\bg}{{\bm g}}
\newcommand{\bh}{{\bm h}}
\newcommand{\bi}{{\bm i}}
\newcommand{\bj}{{\bm j}}
\newcommand{\bk}{{\bm k}}
\newcommand{\bl}{{\bm l}}
\newcommand{\bmm}{{\bm m}}
\newcommand{\bn}{{\bm n}}
\newcommand{\bo}{{\bm o}}
\newcommand{\bp}{{\bm p}}
\newcommand{\bq}{{\bm q}}
\newcommand{\br}{{\bm r}}
\newcommand{\bs}{{\bm s}}
\newcommand{\bt}{{\bm t}}
\newcommand{\bu}{{\bm u}}
\newcommand{\bv}{{\bm v}}
\newcommand{\bw}{{\bm w}}
\newcommand{\bx}{{\bm x}}
\newcommand{\by}{{\bm y}}
\newcommand{\bz}{{\bm z}}
\newcommand{\bA}{{\bm A}}
\newcommand{\bB}{{\bm B}}
\newcommand{\bC}{{\bm C}}
\newcommand{\bD}{{\bm D}}
\newcommand{\bE}{{\bm E}}
\newcommand{\bF}{{\bm F}}
\newcommand{\bG}{{\bm G}}
\newcommand{\bH}{{\bm H}}
\newcommand{\bI}{{\bm I}}
\newcommand{\bJ}{{\bm J}}
\newcommand{\bK}{{\bm K}}
\newcommand{\bL}{{\bm L}}
\newcommand{\bM}{{\bm M}}
\newcommand{\bN}{{\bm N}}
\newcommand{\bO}{{\bm O}}
\newcommand{\bP}{{\bm P}}
\newcommand{\bQ}{{\bm Q}}
\newcommand{\bR}{{\bm R}}
\newcommand{\bS}{{\bm S}}
\newcommand{\bT}{{\bm T}}
\newcommand{\bU}{{\bm U}}
\newcommand{\bV}{{\bm V}}
\newcommand{\bW}{{\bm W}}
\newcommand{\bX}{{\bm X}}
\newcommand{\bY}{{\bm Y}}
\newcommand{\bZ}{{\bm Z}}

% Greek bold letters
\newcommand{\balpha}{{\bm\alpha}}
\newcommand{\bbeta}{{\bm\beta}}
\newcommand{\bgamma}{{\bm\gamma}}
\newcommand{\bdelta}{{\bm\delta}}
\newcommand{\bepsilon}{{\bm\epsilon}}
\newcommand{\bvarepsilon}{{\bm\varepsilon}}
\newcommand{\bzeta}{{\bm\zeta}}
\newcommand{\bfeta}{{\bm\eta}}
\newcommand{\boldeta}{{\bm\eta}}
\newcommand{\btheta}{{\bm\theta}}
\newcommand{\bvartheta}{{\bm\vartheta}}
\newcommand{\biota}{{\bm\iota}}
\newcommand{\bkappa}{{\bm\kappa}}
\newcommand{\blambda}{{\bm\lambda}}
\newcommand{\bmu}{{\bm\mu}}
\newcommand{\bnu}{{\bm\nu}}
\newcommand{\bxi}{{\bm\xi}}
\newcommand{\bpi}{{\bm\pi}}
\newcommand{\bvarpi}{{\bm\varpi}}
\newcommand{\brho}{{\bm\rho}}
\newcommand{\bvarrho}{{\bm\varrho}}
\newcommand{\bsigma}{{\bm\sigma}}
\newcommand{\bvarsigma}{{\bm\varsigma}}
\newcommand{\btau}{{\bm\tau}}
\newcommand{\bupsilon}{{\bm\upsilon}}
\newcommand{\bphi}{{\bm\phi}}
\newcommand{\bvarphi}{{\bm\varphi}}
\newcommand{\bchi}{{\bm\chi}}
\newcommand{\bpsi}{{\bm\psi}}
\newcommand{\bomega}{{\bm\omega}}

\newcommand{\bGamma}{{\bm\Gamma}}
\newcommand{\bDelta}{{\bm\Delta}}
\newcommand{\bTheta}{{\bm\Theta}}
\newcommand{\bLambda}{{\bm\Lambda}}
\newcommand{\bXi}{{\bm\Xi}}
\newcommand{\bPi}{{\bm\Pi}}
\newcommand{\bSigma}{{\bm\Sigma}}
\newcommand{\bUpsilon}{{\bm\Upsilon}}
\newcommand{\bPhi}{{\bm\Phi}}
\newcommand{\bPsi}{{\bm\Psi}}
\newcommand{\bOmega}{{\bm\Omega}}

% Probability and Statistics
\DeclareMathOperator{\Prob}{P}
\DeclareMathOperator{\E}{E}
\DeclareMathOperator{\Var}{Var}
\DeclareMathOperator{\Cov}{Cov}
\DeclareMathOperator{\Corr}{Corr}
\DeclareMathOperator{\sd}{sd}
\DeclareMathOperator{\se}{se}
\DeclareMathOperator{\N}{N}
\DeclareMathOperator{\Bin}{Bin}
\DeclareMathOperator{\Bern}{Bern}
\DeclareMathOperator{\Dir}{Dir}
\DeclareMathOperator{\Wis}{Wis}
\DeclareMathOperator{\logit}{logit}
\DeclareMathOperator{\expit}{expit}
\DeclareMathOperator{\Mult}{Mult}
\DeclareMathOperator{\Cat}{Cat}
\DeclareMathOperator{\Pois}{Poi}
\DeclareMathOperator{\Geom}{Geom}
\DeclareMathOperator{\NBin}{NBin}
\DeclareMathOperator{\Exp}{Exp}
\DeclareMathOperator{\Betadist}{Beta}
\DeclareMathOperator{\Hypergeom}{Hypergeom}
\DeclareMathOperator{\Cauchy}{Cauchy}
\DeclareMathOperator{\hCauchy}{half-Cauchy}
\DeclareMathOperator{\LKJ}{LKJ}
\DeclareMathOperator{\Unif}{Unif}
\DeclareMathOperator{\KL}{KL}
\DeclareMathOperator{\ind}{\mathds{1}}
\newcommand{\iid}{\,\overset{\text{iid}}{\sim}\,}
\DeclareMathOperator*{\plim}{plim}
\DeclareMathOperator{\Lik}{L}
\DeclareMathOperator{\Leb}{Leb}


% Blackboard bold
\newcommand{\bbR}{\mathbb{R}}
\newcommand{\bbN}{\mathbb{N}}
\newcommand{\bbZ}{\mathbb{Z}}
\newcommand{\bbC}{\mathbb{C}}
\newcommand{\bbS}{\mathbb{S}}
\newcommand{\bbH}{\mathbb{H}}
\newcommand{\bbP}{\mathbb{P}}
\newcommand{\bbQ}{\mathbb{Q}}
\newcommand{\bbE}{\mathbb{E}}

% Math calligraphic fonts
\newcommand{\cA}{{\mathcal A}}
\newcommand{\cB}{{\mathcal B}}
\newcommand{\cC}{{\mathcal C}}
\newcommand{\cD}{{\mathcal D}}
\newcommand{\cE}{{\mathcal E}}
\newcommand{\cF}{{\mathcal F}}
\newcommand{\cG}{{\mathcal G}}
\newcommand{\cH}{{\mathcal H}}
\newcommand{\cI}{{\mathcal I}}
\newcommand{\cJ}{{\mathcal J}}
\newcommand{\cK}{{\mathcal K}}
\newcommand{\cL}{{\mathcal L}}
\newcommand{\cM}{{\mathcal M}}
\newcommand{\cN}{{\mathcal N}}
\newcommand{\cO}{{\mathcal O}}
\newcommand{\cP}{{\mathcal P}}
\newcommand{\cQ}{{\mathcal Q}}
\newcommand{\cR}{{\mathcal R}}
\newcommand{\cS}{{\mathcal S}}
\newcommand{\cT}{{\mathcal T}}
\newcommand{\cU}{{\mathcal U}}
\newcommand{\cV}{{\mathcal V}}
\newcommand{\cW}{{\mathcal W}}
\newcommand{\cX}{{\mathcal X}}
\newcommand{\cY}{{\mathcal Y}}
\newcommand{\cZ}{{\mathcal Z}}

% Overbrace and underbrace
\newcommand{\myoverbrace}[3][gray!70]{{\color{#1}\overbrace{\color{black}#2}^{#3}}}
\newcommand{\myunderbrace}[3][gray!70]{{\color{#1}\underbrace{\color{black}#2}_{#3}}}

% Conveniences
\newcommand{\const}{\text{const.}}
\newcommand{\half}[1][1]{\frac{#1}{2}}  % \half for 1/2 or \half[n] for n/2, etc.
\DeclareMathOperator{\diag}{diag}
\DeclareMathOperator{\tr}{tr}
\DeclareMathOperator*{\argmin}{arg\,min}
\DeclareMathOperator*{\argmax}{arg\,max}

% Comments grey text
\newcommand{\mycomment}[2][10pt]{\hspace{#1}\rlap{\color{gray}\text{#2}}}

% Derivatives and integration
\let\d\relax
\DeclareMathOperator{\dd}{d}
\newcommand{\dint}{\dd\hspace{0.5pt}\!}
\newcommand{\d}{\text{d}}

% https://tex.stackexchange.com/questions/19981/how-to-write-rudins-symbol-for-absolute-continuity-of-measures
\DeclareFontFamily{U}{matha}{\hyphenchar\font45}
\DeclareFontShape{U}{matha}{m}{n}{
  <-6> matha5 <6-7> matha6 <7-8> matha7
  <8-9> matha8 <9-10> matha9
  <10-12> matha10 <12-> matha12
  }{}
% \DeclareFontShape{U}{matha}{m}{n}{
%   <5> <6> <7> <8> <9> <10> gen * matha
%   <10.95> matha10 <12> <14.4> <17.28>
%   <20.74> <24.88> matha12
%   }{}

\DeclareSymbolFont{matha}{U}{matha}{m}{n}
\DeclareMathSymbol{\Lt}{3}{matha}{"CE}

\graphicspath{ {figure/} }


\setbeamertemplate{itemize item}[circ]
\setbeamertemplate{itemize subitem}[circ]
\setbeamertemplate{itemize subsubitem}[circ]

\newenvironment{onlyhandout}{\only<handout>{}{}}


\begin{document}

\begin{frame}[plain,noframenumbering]
	\titlepage
\end{frame}

\begin{frame}[plain,noframenumbering]{Overview}
    \tableofcontents
  \end{frame}
\hypertarget{introduction}{%
\section{Introduction}\label{introduction}}

\begin{frame}{Introduction}
The UBD Beamer Theme is a modern and minimal theme designed for getting
information across in a clean and uncluttered manner.

This theme is based on the
\href{https://github.com/kailashbuki/beamerthemesaarland}{Saarland
Beamer Theme}, with its logos and fonts changed, and colour scheme
adapted to UBD's pastel-ised colour scheme.
\end{frame}

\hypertarget{features}{%
\section{Features}\label{features}}

\hypertarget{lists}{%
\subsection{Lists}\label{lists}}

\begin{frame}{Slide full of lists}
\protect\hypertarget{slide-full-of-lists}{}
\emph{Universiti Brunei Darussalam} (UBD; translation University of
Brunei Darussalam; Jawi: \textarabic{يونيبرسيتي بروني دارالسلام}) is the
first university in Brunei.

\begin{itemize}
\tightlist
\item
  UBD in figures

  \begin{itemize}
  \tightlist
  \item
    \textbf{Established}: 1985
  \item
    \textbf{Medium of instruction}: English
  \item
    \textbf{Academic faculties}: 9
  \item
    \textbf{Research Institutes}: 7
  \item
    \textbf{Student enrolment}: 3,137 (in 2015, approx.)
  \end{itemize}
\item
  History

  \begin{itemize}
  \tightlist
  \item
    \textbf{1985}: UBD established, first campus in Gadong
  \item
    \textbf{1995}: UBD moved to Tungku Link
  \item
    \textbf{2009}: Introduction of
    \href{https://ubd.edu.bn/admission/undergraduate/gennext-degree-programme/}{GenNEXT
    Programme}
  \item
    \textbf{2011}: Commencement of the first Discovery Year programme\\
  \end{itemize}
\item
  Credits: \url{https://ubd.edu.bn/} and Wikipedia
\end{itemize}
\end{frame}

\hypertarget{blocks}{%
\subsection{Blocks}\label{blocks}}

\begin{frame}[fragile]{Blocks}
\begin{block}{Standard Block}
\protect\hypertarget{standard-block}{}
This is a standard block using the \texttt{block} environment.
\end{block}

\begin{exampleblock}{Example Block}
This is an example block using the \texttt{exampleblock} environment.

\end{exampleblock}

\begin{alertblock}{Alert Block}
This is an alert block using the \texttt{alertblock} environment.

\end{alertblock}

\begin{altblock}{Alternative Block}
This is an alternatively-coloured block using the \texttt{altblock}
environment.

\end{altblock}
\end{frame}

\hypertarget{quotes}{%
\subsection{Quotes}\label{quotes}}

\begin{frame}{Quotation}
\protect\hypertarget{quotation}{}
\begin{quote}
Archimedes will be remembered when Aeschylus is forgotten, because
languages die and mathematical ideas do not. ``Immortality'' may be a
silly word, but probably a mathematician has the best chance of whatever
it may mean.
\end{quote}

\hfill --- G. H. Hardy in \emph{A Mathematician's Apology, 1941}
\end{frame}

\hypertarget{columns}{%
\subsection{Columns}\label{columns}}

\begin{frame}{Two columns}
\protect\hypertarget{two-columns}{}
We can also add two columns in the slides.

\vspace{1em}

\begin{columns}[T]
\begin{column}{0.48\textwidth}
This is the first column. In this column, we can also add a block for
instance.

\begin{block}{Block}
\protect\hypertarget{block}{}
I am a block in a column.
\end{block}
\end{column}

\begin{column}{0.48\textwidth}
\begin{itemize}
\tightlist
\item
  In this column,
\item
  we just add the
\item
  bullet points.
\end{itemize}
\end{column}
\end{columns}
\end{frame}

\hypertarget{colour-palette}{%
\subsection{Colour palette}\label{colour-palette}}

\begin{frame}[fragile]{Colour palette}
\begin{itemize}
\tightlist
\item
  \textcolor{ubdblue}{Blues: \texttt{ubdblue} a.k.a. Dark Cornflower
  Blue (\#325494)}
\item
  \textcolor{ubdteal}{Teals: \texttt{ubdteal} a.k.a. Upsdell Red
  (\#B10F2E)}
\item
  \textcolor{ubdyellow}{Yellows: \texttt{ubdyellow} a.k.a. Maize Crayola
  Red (\#F5C946)}
\item
  \textcolor{ubdred}{Alerted text: \texttt{ubdred} a.k.a. Upsdell Red
  (\#B10F2E)}
\item
  \textcolor{ubdblack}{Normal text: \texttt{ubdblack} a.k.a. Dark Sienna
  (\#230C0F)}
\item
  \textcolor{gray}{Grays: \texttt{gray} a.k.a. Spanish Gray (\#999999)}
\end{itemize}
\end{frame}

\hypertarget{mathematics}{%
\subsection{Mathematics}\label{mathematics}}

\begin{frame}{Mathematics}
\protect\hypertarget{mathematics-1}{}
Let \(X\sim\operatorname{Pois}(\lambda)\). The probability mass function
of \(X\) is given by \begin{align}\label{eq:pois}
    \Pr(X=x) = \frac{e^{-\lambda}\lambda^x}{x!}.
\end{align} Using the pmf given in \eqref{eq:pois}, we can derive the
moment generating function for \(X\) to be: \begin{align*}
    M_X(t) 
    &= \sum_{k=0}^\infty e^{tx} \cdot \frac{e^{-\lambda}\lambda^x}{x!} \\
    &= e^{-\lambda} \sum_{k=0}^\infty  \frac{(\lambda e^t)^x}{x!} \\
    &= e^{-\lambda}  e^{\lambda e^t} \\
    &= \exp\{\lambda(e^t - 1) \}.
\end{align*}
\end{frame}

\begin{frame}{Theorems et al.}
\protect\hypertarget{theorems-et-al.}{}
\begin{definition}[Prime numbers]
A prime number is a natural number greater than 1 that is not a product
of two smaller natural numbers.
\end{definition}

\begin{theorem}[Infinitude of primes]
There are an infinite number of prime numbers.
\end{theorem}

\begin{proof}
Suppose that there exist only a finite number of primes,
\(p_1,\dots,p_n\), say. The number \[
N = 1+p_1\cdots p_n
\] is divisible by some prime \(p\). But \(p\) cannot be any of
\(p_1,\dots,p_n\), since the latter all leave remainder 1 on dividing
\(N\). This contradicts our assumption that \(p_1,\dots,p_n\) is the
complete list of primes.
\end{proof}
\end{frame}

\begin{frame}{A maths example}
\protect\hypertarget{a-maths-example}{}
Maths examples are continuously numbered (using the \texttt{example}
environment).

\begin{example}[Examples of prime numbers]
2, 3, 5, 7 and 11 are examples of prime numbers.
\end{example}

\begin{example}[Examples of non-prime numbers]
Since \(4 = 2 \times 2\), it is not a prime.
\end{example}
\end{frame}

\hypertarget{citations}{%
\subsection{Citations}\label{citations}}

\begin{frame}{Citations}
The importance of grounding one's self in elementary probability theory
and mathematical statistics cannot be overstated. Here are some
excellent fundamental textbooks every student of statistics should read:
\textcite{casella2002statistical}, \textcite{pawitan2001all}, and
\textcite{wasserman2013all}.

\blfootnote{It is highly suggested to use pandoc's way of generating bibliographies (see \href{https://rmarkdown.rstudio.com/authoring_bibliographies_and_citations.html}{here}) rather than using Biblatex. This footnote was created using the custom \texttt{\textbackslash blfootnote\{\}} command.}
\end{frame}

\hypertarget{using-r}{%
\subsection{\texorpdfstring{Using \texttt{R}}{Using R}}\label{using-r}}

\begin{frame}[fragile]{Syntax highlighting}
\protect\hypertarget{syntax-highlighting}{}
\begin{Shaded}
\begin{Highlighting}[]
\NormalTok{f }\OtherTok{\textless{}{-}} \ControlFlowTok{function}\NormalTok{(x) \{}
  \CommentTok{\# Check small prime}
  \ControlFlowTok{if}\NormalTok{ (x }\SpecialCharTok{\textgreater{}} \DecValTok{10} \SpecialCharTok{||}\NormalTok{ x }\SpecialCharTok{\textless{}} \SpecialCharTok{{-}}\DecValTok{10}\NormalTok{) \{}
    \FunctionTok{stop}\NormalTok{(}\StringTok{"Input too big"}\NormalTok{)}
\NormalTok{  \} }\ControlFlowTok{else} \ControlFlowTok{if}\NormalTok{ (x }\SpecialCharTok{\%in\%} \FunctionTok{c}\NormalTok{(}\DecValTok{2}\NormalTok{, }\DecValTok{3}\NormalTok{, }\DecValTok{5}\NormalTok{, }\DecValTok{7}\NormalTok{)) \{}
    \FunctionTok{cat}\NormalTok{(}\StringTok{"Input is prime!}\SpecialCharTok{\textbackslash{}n}\StringTok{"}\NormalTok{)}
\NormalTok{  \} }\ControlFlowTok{else} \ControlFlowTok{if}\NormalTok{ (x }\SpecialCharTok{\%\%} \DecValTok{2} \SpecialCharTok{==} \DecValTok{0}\NormalTok{) \{}
    \FunctionTok{cat}\NormalTok{(}\StringTok{"Input is even!}\SpecialCharTok{\textbackslash{}n}\StringTok{"}\NormalTok{)}
\NormalTok{  \} }\ControlFlowTok{else} \ControlFlowTok{if}\NormalTok{ (x }\SpecialCharTok{\%\%} \DecValTok{2} \SpecialCharTok{==} \DecValTok{1}\NormalTok{) \{}
    \FunctionTok{cat}\NormalTok{(}\StringTok{"Input is odd!}\SpecialCharTok{\textbackslash{}n}\StringTok{"}\NormalTok{)}
\NormalTok{  \}}
\NormalTok{\}}
\end{Highlighting}
\end{Shaded}
\end{frame}

\begin{frame}[fragile]{Slide with R Output}
\protect\hypertarget{slide-with-r-output}{}
\begin{Shaded}
\begin{Highlighting}[]
\FunctionTok{summary}\NormalTok{(cars)}
\end{Highlighting}
\end{Shaded}

\begin{verbatim}
##      speed           dist       
##  Min.   : 4.0   Min.   :  2.00  
##  1st Qu.:12.0   1st Qu.: 26.00  
##  Median :15.0   Median : 36.00  
##  Mean   :15.4   Mean   : 42.98  
##  3rd Qu.:19.0   3rd Qu.: 56.00  
##  Max.   :25.0   Max.   :120.00
\end{verbatim}
\end{frame}

\begin{frame}[fragile]{Slide with Plot}
\protect\hypertarget{slide-with-plot}{}
\begin{Shaded}
\begin{Highlighting}[]
\FunctionTok{ggplot}\NormalTok{(}\FunctionTok{tibble}\NormalTok{(}\AttributeTok{x =} \FunctionTok{rnorm}\NormalTok{(}\DecValTok{100}\NormalTok{), }\AttributeTok{y =} \FunctionTok{rnorm}\NormalTok{(}\DecValTok{100}\NormalTok{)), }\FunctionTok{aes}\NormalTok{(x, y)) }\SpecialCharTok{+}
  \FunctionTok{geom\_point}\NormalTok{()}
\end{Highlighting}
\end{Shaded}

\begin{center}\includegraphics{figure/pressure-1} \end{center}
\end{frame}

\hypertarget{conclusion}{%
\section{Conclusion}\label{conclusion}}

\begin{frame}[fragile]{Conclusion}
\protect\hypertarget{conclusion-1}{}
To use this theme, download the \texttt{ubd\_beamer\_rmd.tex} file,
\texttt{.sty} files, and image files (for the logo and banner) from
\url{https://github.com/haziqj/ubd-beamer}, and place the files together
with your \texttt{.Rmd} source file. Use the sample
\texttt{slides\_rmd.Rmd} as a guide.
\end{frame}

\begin{frame}[plain,noframenumbering]{End}
	\centering
	\Huge
  \textcolor{ubdblue}{Thank you!}
\end{frame}

\begin{frame}[t,allowframebreaks,noframenumbering,plain]{References}
	\printbibliography[heading=none]
\end{frame}


\end{document}		