\documentclass[]{beamer}

% Options for the UBD theme
\usetheme[
%	progressdots,  % provides progress dots by sections at the top of each slide
	transitions,  % provides transition slides between sections
	banner,  % provides the UBD banner (ubd_banner.png) on the title slide
	logo  % provides the UBD logo (ubd.jpg) in the footers
]{ubd}

% information for the title page
\author{Haziq Jamil}
\title{UBD Beamer Theme}
\subtitle{An unofficial theme for \textit{Universiti Brunei Darussalam}}
\institute{Mathematical Sciences, Faculty of Science, UBD\\ \url{https://haziqj.ml}}
\date{\today}

% Packages 
\usepackage{lipsum}
\usepackage{csquotes}
\usepackage{polyglossia}  % to use arabic
\setdefaultlanguage{english}
\setotherlanguage{arabic}
\newfontfamily\arabicfontsf[Script=Arabic]{Amiri}

% Fix for footnotes not showing when arabic script used
% https://tex.stackexchange.com/questions/228075/beamer-in-arabic-language-doesnt-accept-footnotes
\makeatletter
\let\@footnotetext=\beamer@framefootnotetext
\makeatother

% Bibliography
\usepackage[style=authoryear,giveninits=true,maxcitenames=2,maxbibnames=99,backend=biber,natbib]{biblatex}
\addbibresource{/Users/haziqj/Desktop/GitHub/ubd-beamer/refs.bib}

% Fix URL, DOI, ISBN, etc. font in biblatex
% https://tex.stackexchange.com/questions/416093/change-font-of-the-word-url-before-the-actual-url-in-biblatex
\renewcommand*{\mkbibacro}[1]{#1}  

\begin{document}

\begin{frame}[plain,noframenumbering]
	\titlepage
\end{frame}

\begin{frame}{Overview}
	\tableofcontents
\end{frame}

\section{Introduction}

\begin{frame}{Introduction}

	The UBD Beamer Theme is a modern and minimal theme designed for getting information across in a clean and uncluttered manner.\\[1em]
	
	This theme is based on the \href{https://github.com/kailashbuki/beamerthemesaarland}{Saarland Beamer Theme}, with its logos and fonts changed, and colour scheme adapted to UBD's pastel-ised colour scheme.
	
\end{frame}

\section{Features}
	
\subsection{Lists}

\begin{frame}{Slide full of lists}

	\textit{Universiti Brunei Darussalam} (UBD; translation University of Brunei Darussalam; Jawi: \textarabic{يونيبرسيتي بروني دارالسلام}) is the first university in Brunei.

	\begin{itemize}
		\item UBD in figures
			\begin{itemize}
				\item \textbf{Established}: 1985
				\item \textbf{Medium of instruction}: English
				\item \textbf{Academic faculties}: 9
				\item \textbf{Research Institutes}: 7
				\item \textbf{Student enrolment}: 3,137 (in 2015, approx.)
			\end{itemize}
		\item History
			\begin{itemize}
				\item \textbf{1985}: UBD established, first campus in Gadong
				\item \textbf{1995}: UBD moved to Tungku Link
				\item \textbf{2009}: Introduction of \href{https://ubd.edu.bn/admission/undergraduate/gennext-degree-programme/}{GenNEXT Programme}
				\item \textbf{2011}: Commencement of the first Discovery Year programme			
			\end{itemize}
		\item Credits: \url{https://ubd.edu.bn/} and Wikipedia
	\end{itemize}
\end{frame}

\subsection{Blocks}

\begin{frame}{Blocks}
\framesubtitle{This is a subtitle}

	\begin{block}{Standard Block}
		This is a standard block using the \texttt{block} environment.
	\end{block}
	
	\begin{exampleblock}{Example Block}
		This is an example block using the \texttt{exampleblock} environment.
	\end{exampleblock}
	
	\begin{alertblock}{Alert Block}
		This is an alert block using the \texttt{alertblock} environment.
	\end{alertblock}
	
\end{frame}

\subsection{Quotes}

\begin{frame}{Quotation}

	\begin{quote}
		Archimedes will be remembered when Aeschylus is forgotten, because languages die and mathematical ideas do not. ``Immortality'' may be a silly word, but probably a mathematician has the best chance of whatever it may mean.
	\end{quote}
	\hfill --- G. H. Hardy in \textit{A Mathematician's Apology, 1941}

\end{frame}

\subsection{Columns}

	\begin{frame}{Two Columns}
		We can also add two columns in the slides.
		\begin{columns}[t]
			\begin{column}[T]{0.4\textwidth}
				This is the first column. In this column, we can also add a block for instance.
				\vspace{1em}
				\begin{block}{Block}
					I am a block in a column.
				\end{block}
			\end{column}
			\begin{column}[T]{0.4\textwidth}
				\begin{itemize}
					\item In this column,
					\item we just add the
					\item bullet points.
				\end{itemize}
			\end{column}
		\end{columns}
	\end{frame}

\subsection{Colour palette}

\begin{frame}{Colour palette}

\begin{itemize}
	\item {\color{navyblue} Frame titles: \texttt{navyblue}}
	\item {\color{gray} Structure: \texttt{gray}}
	\item {\color{charcoal} Standard block: \texttt{charcoal}}
	\item {\color{solidpink} Alerted block: \texttt{solidpink}}
	\item {\color{queenpink} Alerted block bg: \texttt{queenpink}}
	\item {\color{myrtlegreen} Example block: \texttt{myrtlegreen}}
	\item {\color{lightcyan} Example block: \texttt{lightcyan}}
	\item {\color{orangecrayola} Misc 1: \texttt{orangecrayola}}
	\item {\color{paradisepink} Misc 2: \texttt{paradisepink}}
\end{itemize}

\end{frame}

\section{Mathematics}

\begin{frame}{Mathematics}

	Let $X\sim\mathrm{Pois}(\lambda)$. 
	The probability mass function of $X$ is given by
	\begin{align}\label{eq:pois}
		\Pr(X=x) = \frac{e^{-\lambda}\lambda^x}{x!}.
	\end{align}
	Using the pmf given in \eqref{eq:pois}, we can derive the moment generating function for $X$ to be:
	\begin{align*}
		M_X(t) 
		&= \sum_{k=0}^\infty e^{tx} \cdot \frac{e^{-\lambda}\lambda^x}{x!} \\
		&= e^{-\lambda} \sum_{k=0}^\infty  \frac{(\lambda e^t)^x}{x!} \\
		&= e^{-\lambda}  e^{\lambda e^t} \\
		&= \exp\{\lambda(e^t - 1) \}.
	\end{align*}

\end{frame}

\begin{frame}{Theorems et al.}

	\begin{definition}[Prime numbers]
		A prime number is a natural number greater than 1 that is not a product of two smaller natural numbers.
	\end{definition}
	
	\begin{theorem}[Infinitude of primes]
		There are an infinite number of prime numbers.
	\end{theorem}
	
	\begin{proof}
		Suppose that there exist only a finite number of primes, $p_1,\dots,p_n$, say.
		The number 
		\[
		  N = 1+p_1\cdots p_n
		\]
		is divisible by some prime $p$.
		But $p$ cannot be any of $p_1,\dots,p_n$, since the latter all leave remainder 1 on dividing $N$.
		This contradicts our assumption that $p_1,\dots,p_n$ is the complete list of primes.
	\end{proof}

\end{frame}

\begin{frame}{A maths example}

	Maths examples are continuously numbered (using the \texttt{example} environment).
	
	\begin{example}[Examples of prime numbers]
		2, 3, 5, 7 and 11 are examples of prime numbers.
	\end{example}

	\begin{example}[Examples of non-prime numbers]
		Since $4 = 2 \times 2$, it is not a prime.
	\end{example}

\end{frame}

\section{Citations}

\begin{frame}{Citations}

	\blfootnote{The \texttt{biblatex} package is highly suggested. This footnote was created using the custom \texttt{\textbackslash blfootnote\{\}} command.}
	
	The importance of grounding one's self in elementary probability theory and mathematical statistics cannot be overstated.
	Here are some excellent fundamental textbooks every student of statistics should read: \citet{casella2002statistical,	pawitan2001all, wasserman2013all}.\\[1em]
	
	\begin{alertblock}{Warning}
		Using fancy tools like neural nets, boosting, and support vector machines without understanding basic statistics is like doing brain surgery before knowing how to use a band-aid \citep{wasserman2013all}.
	\end{alertblock}

\end{frame}

\section{Conclusion}

\begin{frame}{Conclusion}
	To use this theme, download the \texttt{.sty} files and image files (for the logo and banner) from \url{https://github.com/haziqj/ubd-beamer}, and place the files together with your \TeX~source files. Use the sample \texttt{slides.tex} as a guide.
\end{frame}

\begin{frame}[plain,noframenumbering]{End}
	\centering
	\Huge Thank you!
\end{frame}

\begin{frame}[t,allowframebreaks,noframenumbering,plain]{References}
	\printbibliography[heading=none]
\end{frame}

% BACKUP SLIDES -----------------------------------------------------------------------------------------
\appendix
\backupbegin

\section{What this is}

\begin{frame}{Backup slides}

	Often times in a presentation, we don't have enough time to present everything, but it's a good idea to prepare backup slides in case the audience asks about it afterwards.\\[1em]
	
	We can achieve that using the \texttt{\textbackslash appendix} usage.
	
\end{frame}

\section{Backup topic 1}

\begin{frame}{Backup topic 1}

	\lipsum[1]
	
\end{frame}

\section{Backup topic 2}

\begin{frame}{Backup topic 2}

	\begin{block}{A block}
		\lipsum[4]
	\end{block}

\end{frame}

\backupend

\end{document}
